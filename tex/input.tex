\PassOptionsToPackage{unicode=true}{hyperref} % options for packages loaded elsewhere
\PassOptionsToPackage{hyphens}{url}
%
\documentclass[]{article}
\usepackage{lmodern}
\usepackage{amssymb,amsmath}
\usepackage{ifxetex,ifluatex}
\usepackage{fixltx2e} % provides \textsubscript
\ifnum 0\ifxetex 1\fi\ifluatex 1\fi=0 % if pdftex
  \usepackage[T1]{fontenc}
  \usepackage[utf8]{inputenc}
  \usepackage{textcomp} % provides euro and other symbols
\else % if luatex or xelatex
  \usepackage{unicode-math}
  \defaultfontfeatures{Ligatures=TeX,Scale=MatchLowercase}
\fi
% use upquote if available, for straight quotes in verbatim environments
\IfFileExists{upquote.sty}{\usepackage{upquote}}{}
% use microtype if available
\IfFileExists{microtype.sty}{%
\usepackage[]{microtype}
\UseMicrotypeSet[protrusion]{basicmath} % disable protrusion for tt fonts
}{}
\IfFileExists{parskip.sty}{%
\usepackage{parskip}
}{% else
\setlength{\parindent}{0pt}
\setlength{\parskip}{6pt plus 2pt minus 1pt}
}
\usepackage{hyperref}
\hypersetup{
            pdfborder={0 0 0},
            breaklinks=true}
\urlstyle{same}  % don't use monospace font for urls
\setlength{\emergencystretch}{3em}  % prevent overfull lines
\providecommand{\tightlist}{%
  \setlength{\itemsep}{0pt}\setlength{\parskip}{0pt}}
\setcounter{secnumdepth}{0}
% Redefines (sub)paragraphs to behave more like sections
\ifx\paragraph\undefined\else
\let\oldparagraph\paragraph
\renewcommand{\paragraph}[1]{\oldparagraph{#1}\mbox{}}
\fi
\ifx\subparagraph\undefined\else
\let\oldsubparagraph\subparagraph
\renewcommand{\subparagraph}[1]{\oldsubparagraph{#1}\mbox{}}
\fi

% set default figure placement to htbp
\makeatletter
\def\fps@figure{htbp}
\makeatother


\date{}

\begin{document}

\hypertarget{opis}{%
\section{Opis}\label{opis}}

Projekt ten jest implementacją mechanizmu server-push z wykorzystaniem
techniki long-polling, przeznaczoną dla Frameworka Spring. Technika ta
przydaje się w sytuacji, gdy aplikacja kliencka chce pozyskać nowe dane
od serwera - wówczas nie trzeba powtarzać w kółko zapytań otrzymując
puste odpowiedzi, lecz serwer może utrzymywać zapytanie otwarte aż do
pojawienia się nowych danych i dopiero wtedy odesłać odpowiedź.

Zalety takiego podejścia to: - Zmniejszenie liczby niepotrzebnych
zapytań do serwera - Wysoka responsywność aplikacji klienckiej
(aplikacja kliencka otrzymuje dane natychmiast po ich pojawieniu się) -
Bardzo mała ilość zmian wymagana w aplikacji klienckiej (przy przejściu
z ``standardowego'' pollingu do long-pollingu)

Podejście to przejawia wady takie jak: - Zwiększenie obciążenia serwera
- każde otwarte połączenie będzie zwykle absorbowało jeden wątek serwera
(zazwyczaj większość połączeń będzie w bezczynnym stanie oczekiwania)

\hypertarget{opis-implementacji}{%
\subsection{Opis implementacji}\label{opis-implementacji}}

Główny kontroler aplikacji posiada kolejkę obiektów typu ``Promise'' -
reprezentują one początkowo nieznany rezultat jakiegoś działania, który
zostanie ustalony w przyszłości. Wraz z każdym zapytaniem typu
``long-polling'' do owej kolejki dodawany jest nowy obiekt typu
``Promise'' reprezentujący to konkretne zapytanie. Zapytanie pozostaje
otwarte aż do czasu kiedy wartość zostanie ustalona w tymże obiekcie -
kiedy to nastąpi, ustalona wartość zostanie wysłana w ciele odpowiedzi.

Próba ustalenia rezultatu działania w obiektach typu ``Promise''
następuje cyklicznie poprzez wykonanie metody execute() na każdym z
obiektów kolejki. Jeżeli ustalenie rezultatu dla obiektu nastąpiło,
wówczas obiekt jest usuwany z kolejki.

\hypertarget{opis-aplikacji}{%
\subsection{Opis aplikacji}\label{opis-aplikacji}}

Aby zilustrować działanie long-pollingu postanowiliśmy stworzyć
aplikację do wysyłania powiadomień. Po otwarciu aplikacji, wszystkie
powiadomienia są ładowane z bazy danych. Następnie, aplikacja wysyła
zapytanie typu ``long-polling'', aby pozyskać nowe powiadomienia. Jeżeli
nowe powiadomienie pojawi się, wówczas jego szczegóły zostaną zwrócone w
odpowiedzi i aplikacja doda nową notyfikację do widoku aplikacji, po
czym otworzy nowe zapytanie typu ``long-polling'', aby pozyskać kolejne
powiadomienia.

\hypertarget{przykux142ad-uux17cycia}{%
\subsection{Przykład użycia}\label{przykux142ad-uux17cycia}}

Aby użyć stworzonej przez nas biblioteki, kod kliencki musi wykonać
następujące kroki:

\begin{enumerate}
\def\labelenumi{\arabic{enumi}.}
\tightlist
\item
  Kontroler aplikacji musi dziedziczyć po MainController
\end{enumerate}

\begin{verbatim}
@RestController
@RequestMapping("/api")
public class AppController extends MainController {
    ...
}
\end{verbatim}

\begin{enumerate}
\def\labelenumi{\arabic{enumi}.}
\setcounter{enumi}{1}
\tightlist
\item
  Należy zaimplementować własną klasę typu Resolver, której zadaniem
  jest ustalenie wyniku działania dla obiektów typu ``Promise''
\end{enumerate}

\begin{verbatim}
/**
 * Resolver that is successfully resolving Promises, when new record has been added to notification table
 */
@Component
public class NewNotificationResolver implements Resolver, Observer {
    ...
}
\end{verbatim}

\begin{enumerate}
\def\labelenumi{\arabic{enumi}.}
\setcounter{enumi}{2}
\tightlist
\item
  Wraz z żądaniem typu ``long-polling'', do kolejki kontrolera musi
  zostać dodany nowy obiekt typu ``Promise'' reprezentujący żądanie
\end{enumerate}

\begin{verbatim}
@RequestMapping(value = "/newNotification", method = RequestMethod.GET)
public @ResponseBody
DeferredJSON deferredResult() {
    DeferredJSON result = new DeferredJSON(resolver);
    supervisor.add(result);
    return result;
}
\end{verbatim}

\hypertarget{uux17cyte-wzorce-projektowe}{%
\subsection{Użyte wzorce
projektowe:}\label{uux17cyte-wzorce-projektowe}}

\begin{enumerate}
\def\labelenumi{\arabic{enumi}.}
\item
  Data Access Object Pattern

  Poniższe klasy są używane w celu wykonywania operacji na źródle danych
  (np. bazie danych)

  \begin{itemize}
  \tightlist
  \item
    AppUserDao
  \item
    NotificationDao
  \end{itemize}
\item
  Observer Pattern

  \begin{itemize}
  \tightlist
  \item
    NotificationService jest typu Observable - gdy pojawia się nowa
    notyfikacja, powiadamia obserwatorów
  \item
    NewNotificationResolver jest typu Observator - ``rozwiązuje''
    zapytanie asynchroniczne po pojawieniu się nowej notyfikacji
    (wartości w obiekcie typu Promise)
  \end{itemize}
\item
  Command

  Poniższa klasa realizuje wzorzec projektowy ``Command'' gdyż
  enkapsuluje akcję, która jest potrzebna do ``rozwiązania'' zapytania
  asynchronicznego

  \begin{itemize}
  \tightlist
  \item
    DeferredJSON
  \end{itemize}
\item
  Singleton

  Poniższa klasa realizuje wzorzec Singleton, gdyż jest potrzebna
  jedynie jedna instancja połączenia z bazą danych

  \begin{itemize}
  \tightlist
  \item
    DbConnection
  \end{itemize}
\end{enumerate}

\hypertarget{uruchamianie-aplikacji}{%
\section{Uruchamianie aplikacji:}\label{uruchamianie-aplikacji}}

\begin{enumerate}
\def\labelenumi{\arabic{enumi}.}
\tightlist
\item
  W pierwszym kroku należy zaimportować zależności:
\end{enumerate}

\begin{verbatim}
    mvn clean install -U
\end{verbatim}

\begin{enumerate}
\def\labelenumi{\arabic{enumi}.}
\setcounter{enumi}{1}
\tightlist
\item
  Następnie należy wykonać poniższe polecenie:
\end{enumerate}

\begin{verbatim}
    mvn exec:java -Dexec.mainClass="pl.edu.agh.kis.Main" --spring.config.location=classpath:/application.properties 
\end{verbatim}

gdzie \texttt{classpath:/application.properties} oznacza ścieżkę do
pliku konfiguracyjnego aplikacji

\hypertarget{konfiguracja}{%
\section{Konfiguracja}\label{konfiguracja}}

Konfiguracja aplikacji znajduje się w pliku
\texttt{src/main/resources/application.properties}

\hypertarget{build}{%
\section{Build}\label{build}}

Plik wykonywalny .jar ma nazwę \texttt{gs-spring-boot-0.1.0.jar}

\hypertarget{dokumentacja}{%
\subsection{Dokumentacja}\label{dokumentacja}}

\begin{enumerate}
\def\labelenumi{\arabic{enumi}.}
\item
  Javadoc - znajduje się w katalogu javadoc/
\item
  Swagger - znajduje się w pliku ApiDocumentation.pdf lub pod endpointem
  \texttt{/swagger-ui.html} po uruchomieniu aplikacji
\end{enumerate}

\end{document}
